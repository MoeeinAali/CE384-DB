\subsection*{بخش 1}
در این بخش باید همه پیتزافروشی‌هایی را پیدا کنیم که حداقل یک مشتری بالای ۸۰ سال دارند.
\begin{itemize}	
	\item انتخاب افرادی که سن آن ها بالای ۸۰ سال است:
	
	\setLTR
	$Elderly = \sigma_{age \geq 80} (Person)$
	\setRTL
	\item پیدا کردن پیتزافروشی‌هایی که این افراد به آنها مراجعه کرده‌اند:
	
	\setLTR
	$ElderlyFrequents = Elderly \bowtie_{Person.ID=Frequents.personID} Frequents$
	\setRTL
	
	\item جدا کردن نام پیتزافروشی‌ها:
	
	\setLTR
	$Result = \prod_{pizzeria} (ElderlyFrequents)$
	\setRTL
\end{itemize}




\subsection*{بخش 2}
در این بخش باید همه پیتزا فروشی‌هایی را بیابیم که حداقل یک پیتزا را سرو می‌کنند که قیمتی زیر ۲۵۰ هزار
تومن دارد و فردی به نام عموحسن آن را میخورد.

	\setLTR
	$Result = \prod_{pizzeria}(\sigma_{(price<250,000)\land(name="Amoo Hassan")}($
	
	$\qquad \qquad \qquad \quad  Serves\bowtie_{Serves.pizza=Eats.pizza}Eats \bowtie_{Eats.personID=Person.ID}Person))$

	\setRTL


\subsection*{بخش 3}
در این بخش باید پیتزافروشی‌هایی که فقط مشتریان آقا یا فقط مشتریان خانم دارند را پیدا کنیم.
\begin{itemize}	
	\item جدا کردن مشتریان خانم وآقا:
	
	\setLTR
	$MaleCustomers = \sigma_{gender = 'male'} (Person)$
	
	$FemaleCustomers = \sigma_{gender = 'female'} (Person)$
	\setRTL
	
	\item پیدا کردن پیتزافروشی‌هایی که فقط مشتریان آقا را دارند:
	
		\setLTR
$MaleOnly = Frequents - (Frequents \ltimes_{Frequents.personID=MaleCustomers.ID} MaleCustomers)$

$MaleOnlyPizzerias = \prod_{pizzeria} (MaleOnly)$
	\setRTL
	
		\item پیدا کردن پیتزافروشی‌هایی که فقط مشتریان خانم را دارند:
	
	\setLTR
	$FemaleOnly = Frequents - (Frequents \ltimes_{Frequents.personID=FemaleCustomers.ID} FemaleCustomers)$
	
	$FemaleOnlyPizzerias = \prod_{pizzeria} (FemaleOnly)$
	\setRTL
	
	\item پیدا کردن اجتماع پیتزافروشی‌هایی که فقط مشتریان آقا یا فقط مشتریان خانم را دارند:
	
		\setLTR
	$Result = MaleOnlyPizzerias \cup FemaleOnlyPizzerias$
	\setRTL
	
	
\end{itemize}


\subsection*{بخش 4}
در این بخش باید آن افرادی که مشتری همه پیتزافروشی‌هایی که پیتزاهای مورد علاقه آن‌ها را دارد شناسایی کرد.
\begin{itemize}	
	\item
	ابتدا لیست پیتزافروشی‌هایی که حداقل یکی از غذاهای مورد علاقه هر فرد را دارند می‌سازیم:
	
\setLTR
$PizzeriasPerson =\prod_{Pzzeria,personID} (Serves \bowtie_{Serves.pizza =Eats.pizza}Eats)$
\setRTL

	\item 
	در آخر آن افرادی که شامل این Tupple ها نیستند را شناسایی می‎کنیم و از کل افراد حذف می‎کنیم:
	
	\setLTR
	
	$Result = \prod_{personID}Frequents - \prod_{personID}(PizzeriasPerson - Frequents)$
	

	\setRTL
\end{itemize}


\subsection*{بخش 5}
در این بخش باید پیتزافروشی‌هایی را بیابیم که ارزان‌ترین پپرونی‌ها را دارند.
\begin{itemize}
	\item ابتدا ارزان‌ترین پپرونی را می‌یابیم:
	
	\setLTR
	$CheapestPrice = \prod_{price}(Serves) - \prod_{price}[\sigma_{P_1.price>P_2.price}[\rho_{P_1}(Serves)\times\rho_{P_2}(Serves)]]$
	\setRTL
	
	\item حال آن پیتزافروشی‌هایی که این پپرونی را می‌فروشند را پیدا می‌کنیم:
	
	\setLTR
	
	$\prod_{pizzeria}[\sigma_{pizza=pepperoni}(Serves)\bowtie_{Serves.price = CheapestPrice.price}CheapestPrice]$
	\setRTL
\end{itemize}