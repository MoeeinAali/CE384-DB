\subsection*{بخش اول}

\subsubsection*{الف}

\setLTR
$\prod_{Name,Type,Quantity}(\sigma_{(eqquipted=True)\land(Registration Date > 2023)}($

 $\qquad \qquad \qquad User \bowtie_{User.userid=Inventory.PlayerID}Inventory$

 $\qquad \qquad \qquad \qquad  \bowtie_{Inventory.InventoryID = InventoryHasItem.InventoryID}InventoryHasItem$
 
 $\qquad \qquad \qquad \qquad  \bowtie_{InventoryHasItem.ItemID=Item.ItemID}Item))$
\setRTL

\subsubsection*{ب}

\setLTR
$ItemWeapons =\prod_{ItemID} (\sigma_{Type=weapon}(Items))$

$FilteredIDs = \prod_{BuyerID}((\prod_{ItemID,BuyerID}Transactions) \text{\textdiv} (ItemWeapons))$

$Usernames = FilteredIDs \bowtie_{UserIDs.BuyerID = Users.UserID} Users$
\setRTL

\subsubsection*{ج}

\setLTR
$\prod_{TransactionID}[\sigma_{Transactions.Price>100\$}($

$(Transactions\bowtie_{Transactions.BuyerID=Friends.UserID1 \land Transactions.SellerID=Friends.UserID2}Friends)$

$\bigcup$

$(Transactions\bowtie_{Transactions.SellerID=Friends.UserID1 \land Transactions.BuyerID=Friends.UserID2}Friends))]$

\setRTL
	
\subsection*{بخش دوم}

در این کوئری ابتدا آن بازیکن‌هایی که فروشی نداشته‌اند و XP بزرگتر یا مساوی 250 دارند را آیدی‌شان را جدا کرده، سپس کارکترهای آن‌ها را شناسایی کرده، اگر ویژگی آن کارکترها strength بود اسم آن‌ کارکترها را به ما نشان می‌دهد.

\subsection*{بخش سوم}

کوئری B بهتر است و هزینه اجرای کمتری دارد، زیرا ابتدا روی هر رابطه یک بار Selection کرده و سپس ضرب کارتازین را انجام داده ایم، ولی در کوئری A، ابتدا ضرب را انجام داده و سپس Selection را انجام داده‌ایم، این کار باعث می‌شود که محاسبات اضافی زیادی را انجام بدهیم و Table خود را خیلی خیلی بزرگ کنیم در صورتی که به آن نیازی نداریم. 

