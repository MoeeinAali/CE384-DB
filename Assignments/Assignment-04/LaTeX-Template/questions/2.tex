\setLTR
$
R_1 = (X,Y,Z)\\ \\
FDs_1=\begin{cases}
	Y\rightarrow X \\  XZ\rightarrow Y \\ X\rightarrow Z
	\end{cases}
$
\setRTL

ابتدا مجموعه‌ $FDs$ را ساده می‌کنیم:

\setLTR
$
 \longrightarrow FDs_1=\begin{cases}
	Y\rightarrow X \\  X\rightarrow Y \\ X\rightarrow Z
\end{cases} \longrightarrow FDs_1=\begin{cases}
	Y\rightarrow X \\  X\rightarrow YZ 
\end{cases}
$
\setRTL

\subsection*{الف}
این عبارت 
\underline{\textbf{نادرست}}
است.
زیرا $Z$ خودش از $X$ بدست می‌آید و از آن نمی‌توان چیزی را بدست آورد.
\subsection*{ب}
این عبارت 
\underline{\textbf{نادرست}}
است.
اگر رابطه 
$X\rightarrow Z$
را حذف کنیم، از روابط باقی‌مانده نمی‌توان آن را بدست آورد.
\subsection*{ج}
این عبارت 
\underline{\textbf{نادرست}}
است.
زیرا خود $Z$ از $X$ بدست می‌آید. پس وجود $Z$ اضافی است و باید حذف شود.
\subsection*{د}
این عبارت 
\underline{\textbf{درست}}
است.
با توجه به این که دیتابیس ما $1NF$ است و $CK$های ما همگی تک‌عضوی هستند، پس $2NF$ هم هست. با توجه به این که $X,Y$ کلیدهای کاندید ما هستند و ستون $Z$ به ستونی به جز کلیدهای کاندید وابسته نیست، پس دیتابیس $3NF$ هم هست. همچنین در سمت چپ تمامی $FD$های موجود، فقط کلید کاندید وجود دارد، پس دیتابیس ما در نهایت $BCNF$ است.
\subsection*{ه}
این عبارت 
\underline{\textbf{درست}}
است.

\setLTR
$
R_2 = (A,B,C,D,E)\\ \\
FDs_2=\begin{cases}
	A\rightarrow B \\  AB\rightarrow CD \\ D\rightarrow ABC
\end{cases}
$
\setRTL

واضح است که کلیدهای کاندید این دیتابیس، $DE , AE$ هستند. همچنین دیتابیس اول هم دارای کلیدهای کاندید $X,Y$ هستند. پس کلید کاندید دیتابیس حاصل 
$Cartesian \ Product$
این دو، باید شامل یک کلید کاندید از اولی و یک کلید کاندید از دومی باشد. پس در کل $2\times 2$
حالت داریم:


\setLTR
$
CK=\{
YAE,YDE,XAE,XDE
\}
$
\setRTL














