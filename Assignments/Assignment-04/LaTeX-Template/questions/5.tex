\subsection*{الف}
آنومالی‌های رابطه‌ی فعلی:
\begin{enumerate}
	\item INSERT
	
	در رابطه‌ی فعلی اگر بخواهیم یک کتاب اضافه کنیم باید حتما آن را قرض بدهیم و در جدول درج کنیم که این اشتباه است و نوعی آنومالی رایج است. همچنین برای نویسنده‌ هم همچین مشکلی وجود دارد.
	\item UPDATE
	
	همانند سوال قبلی اگر بخواهیم مقدار یکی از ستون‌های جدول را تغییر دهیم‌، باید در تعدادی زیادی رکورد این مقدار را تغییر دهیم که کار هزینه‌بری است.
	
	\item DELETE
	
	همانند سوال قبلی اگر یک سطر را از جدول فعلی حذف کنیم، اطلاعات مربوط به یک نویسنده یا یک کتاب حذف می‌شود که ما این را نمی‌خواهیم.
\end{enumerate}

حال برای تبدیل به $2NF$، اقدام به شکستن 
$Partial \ Dependency$ها 
می‌کنیم:

\setLTR
$
Borrower(\underline{borrowerID},borrowerName) \\ \\
Borrow(\underline{borrowID},dueDate) \\ \\
BorrowBook(\underline{borrowID},\underline{borrowerID},\underline{bookID},) \\ \\
BorrowerFeedback(\underline{borrowerID},\underline{bookID},feedback) \\ \\
Book(\underline{bookID},writer,genre) 
$
\setRTL

اکنون آنومالی مربوط به 
$INSERT,DELETE$
حل شد، حال با تبدیل به $3NF$ آنومالی مربوط به $UPDATE$ را حل می‌کنیم:

\setLTR
$
bookWriter(\underline{bookID},writer) \\ \\
writerGenre(\underline{Writer,Genre}) \\ \\
Borrower(\underline{borrowerID},borrowerName) \\ \\
Borrow(\underline{borrowID},dueDate) \\ \\
BorrowBook(\underline{borrowID},\underline{borrowerID},\underline{bookID},) \\ \\
BorrowerFeedback(\underline{borrowerID},\underline{bookID},feedback) \\ 
$
\setRTL

\subsection*{ب}

هیچ کدام از وابستگی‌ها از بین نرفته است، پس هردو دارای 
$Dependency \ Preserving$
هستند.
\pagebreak 

\subsection*{ج}

همچنان برای $INSERT,DELETE$ آنومالی وجود دارد، به عنوان مثال برای حذف یک کتاب مجبوریم که تمامی فیدبک‌های آن را یکی‌یکی حذف کنیم.

حال برای تبدیل به $BCNF$، رابطه‌ی $BorrowFeedback$ را می‌شکنیم:

\setLTR
$
BorrowFeedback(\underline{borrowerID},Feedback)\\ \\
BookFeedback(\underline{Feedback},bookID)
$
\setRTL


با اضافه کردن دو جدول فوق، رابطه‌ی 

$(BorrowerID,BookID)\rightarrow FeedBack$
از بین رفت و 
$Dependency \ Preserving$
نقض شد.

\subsection*{د}
به عنوان مثال اگر در جدول $BorrowBook$ به جای استفاده از $PK$، از $BookID$ استفاده کنیم و جوین بزنیم، آن‌گاه اگر یک کتاب خاص چند بار به افراد متنوع قرض داده شده باشد، دیتاهایی را از دست میدهیم و دیگر $LossLess$ نیست!


