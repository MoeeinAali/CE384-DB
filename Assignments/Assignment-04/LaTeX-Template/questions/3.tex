\subsection*{الف}
مراحل زیر را طی می‌کنیم:
\begin{enumerate}
	\item 
	هر وابستگی تابعی در F را به طوری تجزیه کنید که در سمت راست فقط یک ستون وجود داشته باشد.
	\item 
صفات اضافی را با محاسبه بسته closure صفات سمت چپ به جز صفت مورد نظر، حذف کنید.
\item 
وابستگی‌های تابعی زائد را حذف کنید.
\item
اطمینان حاصل کنید که سمت راست هر وابستگی تابعی فقط شامل یک صفت باشد.
\item 
روابطی که با استفاده از تعدی به وجود آمده‌اند را حذف می‌کنیم.
\end{enumerate}

\subsection*{ب}

پس از اجرای الگوریتم فوق، به مجموعه $FDs$ زیر می‌رسیم:

\setLTR
$
FDs = \begin{cases}
msgID,wordPosition \rightarrow wordID \\
wordID \rightarrow wordText \\
wordText \rightarrow wordID \\
msgID \rightarrow  visibility\\
msgID \rightarrow  userID\\
\end{cases}
$
\setRTL


\subsection*{ج}
این عبارت 
\underline{\textbf{درست}}
است. زیرا:

\setLTR
$
\begin{cases}
	msgID,wordID \rightarrow visibility \\
	msgID \rightarrow userID
\end{cases} \longrightarrow 
\begin{cases}
	msgID,wordID,visibility \rightarrow userID
\end{cases}
$
\setRTL


\subsection*{د}
می‌دانیم که کلید کاندید ما 
$\{msgID,wordPosition\}$
است. در ابتدا 
$Partial \ Dependency$
موجود در $wordID$ و سپس 
$Transitive \ Dependency$
موجود بین 
$wordText$
و کلید کاندید را از بین می‌بریم. پس در کل به 4 رابطه نیاز داریم:

\setLTR
$
R_1 = \{\underline{msgID},visibility\}\\
R_2 = \{\underline{msgID},userID\} \\
R_3 = \{\underline{msgID,wordPosition},wordID\}\\
R_4 = \{\underline{wordID},wordText\}
$
\setRTL

با توجه به اینکه که مجموعه کلیدهای کاندید ما تک عضوی است، پس دیتابیس ما 
$BCNF$
هم هست.

در نتیجه چون نرمال‌سازی ما در سطح $BCNF$ است، پس $LossLess$ است و $Dependency$ها را حفظ می‌کند.
\subsection*{ه}

\setLTR
$
R = R_1 \bowtie_{msgID} R_2 \bowtie_{msgID} R_3 \bowtie_{wordID} R_4
$
\setRTL
