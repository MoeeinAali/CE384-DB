\subsection*{الف}
\begin{enumerate}
	\item INSERT
	
	به عنوان مثال، ستون‌های $itemID$ و $memberID$ و... فقط در سمت راست $FD$ مربوط به امانات وجود دارند که این به معنی آن است که برای درج کردن یک عضو یا یک کالا یا نویسنده یا ...، حتما باید رابطه امانت گرفتن و یا نویسندگی برای آن‌ها برقرار باشد.
	\item DELETE
	
	همان مواردی که در قسمت قبل برای $INSERT$ مشکل ایجاد کرده بودند، برای حذف کردن هم مشکلاتی به وجود می‌آورند. به عنوان مثال اگر یک رکورد از امانت‌ها را حذف کنیم، کالا یا عضو یا... مربوط به آن هم ممکن است به کل از دیتابیس حذف شود اگر در رکورد دیگری وجود نداشته باشد.
	\item UPDATE
	
اگر بخواهیم ستون‌های سمت راست $FD$ها را تغییر دهیم، نیازمند این هستیم که تمامی رکوردهای مربوط به آن مقدار خاص از یک ستون را تغییر دهیم که پروسه‌ی هزینه‌بر و زمان‌بری است.
\end{enumerate}









\subsection*{ب}
می‌دانیم که در رابطه‌ی فعلی، ستون 
$loanID$
تنها PrimaryKey و تنها CandidateKey است. پس این رابطه
$2NF$
است. اما چون در روابطمان خاصیت تعدی را داریم، باید آن را به 
$3NF$
تبدیل کنیم و ستون‌ها را بشکنیم.

رابطه‌های جدید به این صورت هستند:

\setLTR
$
R_1 = \{\underline{memberID},memberName,memberAddress,memberType\} \\ 
R_2 = \{\underline{loanID},loanDate,dueDate,returnDate,lateDate,memberID,itemID\} \\
R_3 = \{\underline{itemID},itemTitle,itemType,itemFormat,authorID\} \\
R_4 = \{\underline{authorID},authorName\} \\
R_5 = \{\underline{itemType,memberType},lateFee\}
$
\setRTL

حال چون تمامی روابط فوق به صورت 
$PK\rightarrow X$
است، پس دیتابیس فوق به صورت 
$BCNF$
نیز می‌باشد و نیاز به تغییر ندارد.



\subsection*{ج}

در طراحی جدید می‌توان به راحتی موارد جدید را افزود. به عنوان مثال اگر بخواهیم $item$ جدیدی را اضافه کنیم، فقط کافیست آن را به جدول $items$ اضافه کنیم.

همچنین اگر بخواهیم یک ویژگی جدید به یک موجودیت اضافه کنیم، کافیست که فقط جدول مربوط به آن را تغییر دهیم و روابط دیگر دچار تغییر نمی‌شوند.

اگر بخواهیم یک رابطه جدید هم تعریف کنیم، تنها روابطی که با آن‌ها درگیر هستند نیازمند تغییر و اضافه کردن 
$FK$
هستند، نه همه‌ی روابط!

خلاصه که آنومالی‌های قسمت الف برطرف می‌شوند.








