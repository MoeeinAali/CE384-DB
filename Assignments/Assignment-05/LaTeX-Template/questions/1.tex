\subsection*{الف}
Collection Capped در MongoDB یک مجموعه با ظرفیت ثابت است که داده‌ها به ترتیب وارد شدن در آن ذخیره می‌شوند. در صورت تجاوز تعداد داده‌ها از حد تعیین‌شده، داده‌های قدیمی‌تر حذف شده و داده‌های جدید جایگزین آن‌ها می‌شوند. این ویژگی برای سناریوهایی مناسب است که نیاز به نگهداری داده‌های جدید دارند و داده‌های قدیمی‌تر اهمیت چندانی ندارند.
\subsection*{ب}

\subsubsection*{مزایا}
\begin{itemize}
	\item مدیریت خودکار حذف داده‌های قدیمی:
	وقتی که حجم مجموعه به حداکثر ظرفیت خود برسد، داده‌های قدیمی به صورت خودکار حذف می‌شوند تا جا برای داده‌های جدید فراهم شود. این ویژگی برای کاربردهایی که نیاز به نگهداری فقط آخرین n داده دارند، بسیار مناسب است.
	\item حفظ نظم ورود داده‌ها:
	اسناد در Collections Capped به ترتیب ورود آنها ذخیره می‌شوند و این ویژگی می‌تواند در کاربردهایی که نیاز به نگهداری ترتیب زمانی داده‌ها دارند، بسیار مفید باشد.
	\item زمان ثابت برای جستجو:
	به دلیل محدود بودن اندازه و تعداد اسناد، زمان جستجو و بازیابی داده‌ها ثابت و بهینه است.
	\item درج سریع‌تر:
	Collections Capped به دلیل اینکه اندازه آنها از قبل تعیین شده است و فضای دیسک مورد نیاز از ابتدا تخصیص یافته است، عملیات درج داده‌ها بسیار سریع انجام می‌شود.
\end{itemize}
\subsubsection*{معایب}
\begin{itemize}
	\item عدم امکان حذف اسناد خاص:
	نمی‌توان اسناد خاصی را به صورت انتخابی حذف کرد. فقط اسناد قدیمی‌تر به صورت خودکار حذف می‌شوند
	\item فضای از پیش تخصیص داده شده:
	اندازه یک Collection Capped از ابتدا مشخص می‌شود و نمی‌توان آن را افزایش داد. بنابراین، در صورتی که نیاز به نگهداری داده‌های بیشتر از ظرفیت تعیین شده باشد، این محدودیت می‌تواند مشکل‌ساز باشد.
	\item عدم انعطاف‌پذیری در تغییر اندازه: 
	اگر نیاز به تغییر اندازه مجموعه باشد، باید مجموعه جدیدی با اندازه جدید ایجاد کرده و داده‌ها را به آن منتقل کرد.
	\item عدم امکان به‌روزرسانی اندازه اسناد:
	اسنادی که وارد یک Collection Capped می‌شوند، باید اندازه ثابتی داشته باشند. اگر اندازه یک سند به‌روزرسانی شود و از اندازه اولیه بیشتر شود، عملیات به‌روزرسانی ممکن است شکست بخورد.
	\item عدم پشتیبانی از برخی ایندکس‌ها
\end{itemize}
\pagebreak
\subsection*{ج}
استفاده از Collections Capped در MongoDB در سناریوهایی توصیه می‌شود که نیاز به مدیریت داده‌ها به صورت ترتیبی و با حجم محدود دارند و نیاز به ذخیره دادە‌های جدید و تازە‌تر داریم و دادە‌های قدیمی‌تر
برای ما اهمیت ندارند. همچنین به عنوان مثال در ⅽaⅽhing نیز کاربرد دارد.
\textbf{چند مثال عملی:}
\begin{itemize}
	\item ⅽaⅽhing
	
	فرض کنید یک برنامه وب دارید که کاربران می‌توانند در آن جستجو کنند و نتایج جستجو از یک منبع داده پیچیده و زمان‌بر (مثل یک پایگاه داده بزرگ یا یک API خارجی) گرفته می‌شود. برای افزایش کارایی و کاهش زمان پاسخگویی، می‌توانید نتایج جستجوهای اخیر را cache کنید تا در صورت درخواست مجدد همان جستجو، نتایج به سرعت از cache بازیابی شوند.
	\item سیستم‌های لاگینگ:

	سرورهایی که لاگ‌های رخدادها (event logs) را ذخیره می‌کنند، نیاز به نگهداری لاگ‌های اخیر دارند و لاگ‌های قدیمی‌تر به مرور زمان حذف می‌شوند. با استفاده از Collections Capped، لاگ‌های جدید به سرعت اضافه می‌شوند و لاگ‌های قدیمی به صورت خودکار حذف می‌شوند.
	\item سیستم‌های صف‌:
	
	صف‌های پیام (message queues) که پیام‌های ورودی را به ترتیب دریافت و پردازش می‌کنند. در این سیستم‌ها، ممکن است فقط نیاز به نگهداری تعداد محدودی از پیام‌های اخیر باشد.
		
\end{itemize}

\subsection*{د}

فرض کنید می‌خواهیم یک Collection Capped به نام logs با حداکثر اندازه 10 مگابایت و حداکثر 1000 سند ایجاد کنیم:
\setLTR
\begin{lstlisting}
db.createCollection("logs", {
	capped: true,
	size: 10485760,  // 10MB
	max: 1000        // 1000 Record
});
\end{lstlisting}
\setRTL

:capped
 این پارامتر باید true باشد تا مجموعه به صورت Capped ایجاد شود.

:size
 اندازه کلی مجموعه را به بایت مشخص می‌کند. این پارامتر تعیین می‌کند که مجموعه چه مقدار فضا در دیسک اشغال می‌کند.
 
 :max
 حداکثر تعداد اسنادی که می‌توانند در مجموعه قرار گیرند. اگر این پارامتر تنظیم شود، حتی اگر اندازه مجموعه به مقدار size نرسد، با رسیدن به این تعداد سند، اسناد قدیمی حذف خواهند شد.
 
 \textbf{تاثیر بر روی عملکرد}
 \begin{itemize}
 	\item درج سریع‌تر:
 	به دلیل از پیش تخصیص داده شدن فضا و عدم نیاز به افزایش اندازه مجموعه، عملیات درج داده‌ها در Collection Capped به طور قابل توجهی سریع‌تر است.
 	\item  زمان ثابت برای جستجو:
 	با توجه به محدود بودن اندازه مجموعه، زمان جستجو و بازیابی داده‌ها ثابت و بهینه است.
 \end{itemize}
  \textbf{تاثیر بر روی مدیریت داده}
   \begin{itemize}
  	\item حذف خودکار داده‌های قدیمی
  	\item محدودیت در حذف انتخابی
  	\item عدم امکان به‌روزرسانی اندازه اسناد
  	
  \end{itemize}