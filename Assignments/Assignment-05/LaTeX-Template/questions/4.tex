این کوئری همه افرادی که 20 سال سن دارند و از بین دوستان آن‌ها حداقل یک نفر به نام $Jane \ Doe$ وجود دارد را برمی‌گرداند:
\setLTR
\begin{lstlisting}
db.students.find(
	{"age":20, "friends.name": "Jane Doe"}
)
\end{lstlisting}
\setRTL


این کوئری افرادی را می‌دهد که تاریخ اد شدن آن‌ها بزرگتر از تاریخ داده شده باشد:
\setLTR
\begin{lstlisting}
db.students.find(
	{"createdAt": { $gt: ISODate("2021-01-01T00:00:00Z")}}
)
\end{lstlisting}
\setRTL


این کوئری به ما افرادی را برمی‌گرداند که در آنها address.city برابر با "Anytown" است و فقط فیلدهای name و email از آن افراد نمایش داده می‌شوند، در حالی که id پنهان خواهد بود:
\setLTR
\begin{lstlisting}
db.students.find(
	{"address.city": "Anytown"}, 
	{"name": 1, "email": 1, "_id": 0}
)
\end{lstlisting}
\setRTL

این کوئری افرادی را می‌دهد که نام آن‌ها با J شروع می‌شود:
\setLTR
\begin{lstlisting}
db.students.find(
	{"name": {$regex: /^J/}}
)
\end{lstlisting}
\setRTL

این کوئری افرادی  را که مقدار فیلد address.state آنها "CA" است، پیدا می‌کند و سپس نتایج را بر اساس فیلد age به ترتیب صعودی سورت می‌کند:
\setLTR
\begin{lstlisting}
db.students.find(
	{"address.state": "CA"}).sort({"age": 1}
)
\end{lstlisting}
\setRTL


این کوئری جوان‌ترین فرد را نشان می‌دهد:
\setLTR
\begin{lstlisting}
db.students.find().sort({"age": 1}).limit(1)
\end{lstlisting}
\setRTL


این کوئری افرادی را که ایمیل آن‌ها ثبت نشده است را نشان می‌دهد:
\setLTR
\begin{lstlisting}
db.students.find(
	{"email": {$exists: false}}
)
\end{lstlisting}
\setRTL

این کوئری افرادی را که بیش از 1 دوست دارند را نشان می‌دهد:
\setLTR
\begin{lstlisting}
db.students.find({
	"friends": {$size: {$gt: 1}}}
)
\end{lstlisting}
\setRTL

این کوئری افرادی را نشان می‌دهد که بین 20 تا 30 سال سن دارند:
\setLTR
\begin{lstlisting}
 db.students.find(
 	{"age": {$gte:20, $lte:30}}
)
\end{lstlisting}
\setRTL
\pagebreak

این کوئری گروه بندی خاصی انجام نمیدهد زیرا ID ما برابر با Null است، درنتیجه در اصل میانگین سن همه افراد را دارد محاسبه می‌کند:
\setLTR
\begin{lstlisting}
db.students.aggregate([{
	$group: {
		_id: null,averageAge: {$avg: "$age"}
	}
}])
\end{lstlisting}
\setRTL

این کوئری رکوردها را با فیلدهای جدیدی خروجی می‌دهد. نام و ایمیل را از قبل نشان می‌دهد و دو فیلد تعداد دوستان و طول اسم را جدید اضافه می‌کند:
\setLTR
\begin{lstlisting}
db.students.aggregate(
[{
	$project: {name: 1,
			   email: 1,
			   numberOfFriends: {$size: "$friends"},
			   nameLength: {$strLenCP: "$name"}
			  }
}])
\end{lstlisting}
\setRTL
